\chapter{Conclusions}

While the ultimate goal of constructing an artificial general intelligence still seems to be way in the distance, machine learning has been advancing quickly towards solving more and more advanced problems, with no end in sight for the current wave of progress. However, the field has known setbacks and periods of little activity, and there is no guarantee such bottlenecks will not occur again.

Fortunately, nowadays we know a lot more about intelligence and the human brain than we did over 50 years ago. We have rigorous theoretical frameworks for the problem of learning, allowing us to understand the trade-offs involved in constructing machines capable of improving based on previous experience.

It is my hope that the results presented here showcase the importance of performing a thorough rigorous analysis of the problem at hand, before jumping to solving it and evaluating the empirical performance of the solution. The limit of how powerful the intelligent machines we construct isn't only computational power or the availability of large data sets, but also the intrinsic difficulty of the concepts we are trying to learn and the bounded capacity of our models.

% TODO: add an extra idea or two here
